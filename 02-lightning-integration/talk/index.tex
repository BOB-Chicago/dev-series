\documentclass{beamer}
\title{Introduction to lightning network protocol ideas}
\author{Ian Shipman}
\date{BOB Chicago {workshop} June 26, 2018}
\begin{document}
\frame{\titlepage}
\begin{frame}
	\frametitle{A trustful payment channel}
	Alfred and Brenda know and trust each other.  The create, but don't publish, 
	sequence of transactions:
	\begin{itemize}
		\item (TX1: channel open) \\ 
			Alfred (in: 0.1, out: 0.1) \\
			Brenda (in: 0.1, out: 0.1)
		\item (TX2: Brenda pays for a cab ride) \\
			Delete TX1 \\
			Alfred (in: 0.1, out: 0.102) \\
			Brenda (in 0.1, out: 0.098)
		\item (... TXN: Alfred pays for movie tickets) \\
			Delete TX(N-1) \\ 
			Alfred (in: 0.1, out: 0.149) \\
			Brenda (in: 0.1, out: 0.059)
	\end{itemize}
	Time to settle up!  Alfred \& Brenda publish (TXN).
\end{frame}
\begin{frame}
	\frametitle{Opportunities to cheat}
	\begin{enumerate}
		\item Alfred or Brenda could spend one of the outputs they are using to 
			fund their transactions.
		\item At any point before they settle up, one of them can 
			publish an old transaction.  To a third party on the network, their 
			various channel states are \emph{all equally valid}.  
	\end{enumerate}
	\pause
	\begin{block}{Question}
		How can Alfred \& Brenda create the sequence of transactions in such a way 
		that cheating is much more difficult?
	\end{block}
	\pause
	Verifiable deletion is not possible.  There has to be a way either 
	\emph{override} a previous transaction or \emph{penalize} its use.
\end{frame}
\begin{frame}
	Now a quick digression into transaction structure
\end{frame}
\begin{frame}
	\frametitle{Pre-SegWit transaction structure}
	\begin{block}{Inputs}
		\begin{itemize}
			\item reference to a previous transaction \pause
			\item output index \pause
			\item scriptSig \pause
		\end{itemize}
	\end{block}
	\begin{block}{Outputs}
		\begin{itemize}
			\item amount \pause
			\item redeemScript \pause 
		\end{itemize}
	\end{block}
	\begin{example}[P2KH script]
		\begin{itemize}
			\item (redeemScript) {\tiny \texttt{OP\_DUP OP\_HASH160 \$recipientKeyHash OP\_EQUALVERIFY OP\_CHECKSIG}}
			\item (scriptSig) {\tiny\texttt{\$signature \$publicKey}}
		\end{itemize}
	\end{example}
\end{frame}
\begin{frame}
	\frametitle{SegWit transaction structure}
	\begin{block}{Inputs} 
		\begin{itemize}
			\item scriptSig: empty (native) or {\tiny \texttt{OP\_0 \$redeemScriptHash}} \pause
		\end{itemize}
	\end{block}
	\begin{block}{Outputs}
		... same as before for our purposes \pause
	\end{block}
	% Take away: transaction ID can be known before the tx is signed
	\begin{block}{Take away}
		\begin{itemize}
			\item The transaction ID is known \emph{even without} the witness (signature) data. \pause
			\item Alfred and Brenda can now safely lock funds into a channel.
		\end{itemize}
	\end{block}
\end{frame}
\begin{frame}
	\frametitle{Opening a channel}
	\begin{enumerate}
		\item Alfred and Brenda each contribute a SegWit input to a transaction 
			which has a single 2-of-2 multisig output, but do not share their witness 
			data \pause
		\item They construct the first state update in some way referring to the 
			opening transaction and contribute their signatures \pause
		\item With the refunding transaction(s) safely in hand, they contribute 
			their witness data to the funding transaction  \pause
		\item They publish the funding transaction, locking funds into the channel
	\end{enumerate}
\end{frame}
% A time based solution
\begin{frame}
	\frametitle{Leveraging timelocks}
	Alfred and Brenda can improve their channel by adding timelocks.
	\pause
	\begin{itemize}
		\item They agree on a channel expiration $ T = 20000 $ \pause
		\item The $ n^{th} $ channel update is locked for $ T - 10 n $ blocks. \pause
		\item When they are ready to settle up, they create and sign a version of 
			the most recent transaction without the timelock and publish it. \pause
	\end{itemize}
	\begin{block}{If somebody cheats ...}
		Suppose Alfred publishes a stale transaction from the sequence.  Then Brenda 
		can publish the most recent state and then spend her output before Alfred's 
		time lock expires.  Unfortunately it can take \emph{a long time} to settle 
		the dispute.
	\end{block}
	\pause
	\begin{block}{But if Brenda goes offline...}
		... then she will not see that the old state was published and may not be 
		able to intervene in time.
	\end{block}
\end{frame}
\begin{frame}
	\frametitle{Leveraging hashlocks}
	Another approach uses penalties to revoke old channel states. \pause
	A \emph{hash lock} is the bitcoin script fragment \\
	{\tiny \texttt{OP\_SOMEHASH \$targetHash OP\_EQUALVERIFY}} \\ \pause
	To make this bit of script succeed, you must know \emph{and publish to the 
	world} the preimage of \texttt{\$targetHash} \pause
	\begin{block}{Pairs of channel updates}
		Alfred generates a random secret $ s $ and prepares a 
		transaction expressing the current state with outputs 
		\pause
		\begin{itemize}
			\item (OA) Alfred can redeem in $ N $ blocks OR Brenda can redeem if she 
				knows $ s $  \pause
			\item (OB) Brenda can redeem immediately \pause
		\end{itemize}
		Alfred:  (unsigned!) transaction + \emph{previous} secret $ s' $ $ \Rightarrow $ Brenda \pause
		Brenda: if $ s' $ unlocks the previous output for her, signature $ \Rightarrow $ Alfred
	\end{block}
\end{frame}
\begin{frame}
	\frametitle{Penalty}
	At any point each party could have \pause
	\begin{enumerate}
		\item A series of their own (stale) state updates, signed by their 
			counterparty \pause
		\item The secrets that unlock their counterparty's output on each of the 
			counterparty's old state updates \pause
	\end{enumerate}
	To be safe each person should delete the old states! \pause
	\begin{block}{Cheating}
	Suppose Alfred attempts to cheat by publishing an old state \\ \pause
	He has to wait to redeem his output... \\ \pause
	... but Brenda can immediately take both outputs for herself because she 
		knows the secret associated with Alfred's output on the old state
	\end{block}
\end{frame}
\begin{frame}
	\frametitle{Forming a network}
	\begin{itemize}
		\item Alfred and Brenda can pay each other as long as they both have an 
			open channel with Chelsea \pause
		\item Before Brenda updates her channel with Chelsea, she needs a guarantee 
			that Chelsea will correspondingly update her channel with Alfred \pause
	\end{itemize}
	\begin{block}{Secret revelation again}
		\begin{itemize}
			\item Chelsea should need to know a secret to redeem her new balance from Brenda and 
				she should have to pay Alfred for this secret \pause
			\item Alfred must not be able to punish Chelsea by witholding the secret 
		\end{itemize}
	\end{block}
\end{frame}
\begin{frame}
	\frametitle{Protocol}
	\begin{enumerate}
		\item Alfred generates a route secret and sends its hash to Brenda as part of an invoice \pause
		\item Brenda and Chelsea work out a channel update to their channel which allows Alfred to 
			spend Chelsea's output before Chelsea by publishing the route secret but after a certain delay \pause
		\item Alfred and Chelsea work out a channel update to their channel where 
			Alfred must reveal the route secret to spend his output \pause
	\end{enumerate}
	\begin{block}{Notes}
		\begin{itemize}
			\item Brenda and Chelsea can freely share their channel updates with 
				Alfred for verification since they are unsigned	\pause
			\item The channel updates still have the same features as before \emph{in 
				addition} to the hash lock that protects the route 
		\end{itemize}
	\end{block}
\end{frame}
\end{document}
